
%=== use the these packages if they are not already in use ===
%\usepackage{amsfonts}
%\usepackage{amsmath}
%\usepackage{amssymb}
\usepackage{amsthm}
\usepackage{bbm} 
%\usepackage{cite}
\usepackage{color}
%\usepackage{euscript}
\usepackage{graphicx}
\usepackage{mathrsfs} 
%\usepackage{microtype}
% \usepackage[normalem]{ulem}
% \usepackage{wrapfig}
%=============================================================

%===== control =====
\allowdisplaybreaks

%===== fonts =====
\def\ttt{\texttt}
\def\tsc{\textsc}

%===== spacing =====

\def\extraspacing{\vspace{3mm} \noindent}
\def\figcapup{\vspace{-1mm}}
\def\figcapdown{\vspace{-0mm}}
\def\hgap{\textrm{\hspace{1mm}}}
\def\thmvgap{\vspace{0mm}}
\def\vgap{\vspace{1mm}}


%===== tabbing =====

\def\tab{\hspace{3mm}}
\def\tabpos{\hspace{4mm} \= \hspace{4mm} \= \hspace{4mm} \= \hspace{4mm} \= \hspace{4mm} \= \hspace{4mm} \= \hspace{4mm} \= \hspace{4mm} \= \hspace{4mm} \= \hspace{4mm} \= \hspace{4mm} \= \hspace{4mm} \= \hspace{4mm} \= \hspace{4mm} %\= 
%\= \hspace{4mm} \= \hspace{4mm} \= \hspace{4mm} \= \hspace{4mm} \= \hspace{4mm} \= \hspace{4mm}
\kill}
\newcommand{\mytab}[1]{\begin{tabbing}\tabpos #1\end{tabbing}}

%===== blocks =====

%%%%%%%%%%%%%%%%%%%%%%%%%%%%%%%%
% THEOREMS
%%%%%%%%%%%%%%%%%%%%%%%%%%%%%%%%
% \theoremstyle{plain}
% \newtheorem{theorem}{Theorem}[section]
% \newtheorem{proposition}[theorem]{Proposition}
% \newtheorem{lemma}[theorem]{Lemma}
% \newtheorem{corollary}[theorem]{Corollary}
% \theoremstyle{definition}
% \newtheorem{definition}[theorem]{Definition}
% \newtheorem{assumption}[theorem]{Assumption}
% \theoremstyle{remark}
% \newtheorem{remark}[theorem]{Remark}

% \newtheorem{theorem}{Theorem}
% \newtheorem{lemma}[theorem]{Lemma}
% \newtheorem{corollary}[theorem]{Corollary}
% \newtheorem{proposition}{Proposition}
% \newtheorem{definition}{Definition}
% \newtheorem{problem}{Problem}

\newcommand{\boxminipg}[2]{\begin{center}\fbox{\begin{minipage}{#1}#2\end{minipage}}\end{center}}
\newcommand{\minipg}[2]{\begin{center}\begin{minipage}{#1}#2\end{minipage}\end{center}}
\newcommand{\myitems}[1]{\begin{itemize} #1 \end{itemize}}
\newcommand{\myenums}[1]{\begin{enumerate} #1 \end{enumerate}}
\newcommand{\myfig}[1]{\begin{figure}\centering #1\end{figure}}
\newcommand{\myfigg}[2]{\begin{figure}\centering #1 \figcapup \caption{#2} \figcapdown \end{figure}}
\newcommand{\myfigstar}[2]{\begin{figure*}\centering #1 \figcapup \caption{#2} \figcapdown \end{figure*}}

%===== math macros =====

\newcommand{\bm}[1]{\textrm{\boldmath${#1}$}}
\newcommand{\mb}[1]{\mathbf{#1}}
% \newcommand{\smat}[2]{\left[\begin{tabular}{#1}#2\end{tabular}\right]}
% \newcommand{\bmat}[2]{\left|\begin{tabular}{#1}#2\end{tabular}\right|}
\newcommand{\bmat}[1]{\begin{bmatrix}#1\end{bmatrix}}
\newcommand{\vmat}[1]{\begin{vmatrix}#1\end{vmatrix}}
\newcommand{\myeqn}[1]{\begin{eqnarray}#1\end{eqnarray}}
\newcommand{\myset}[1]{\{#1\}}
\newcommand{\set}[1]{\{#1\}}

\newcommand{\explain}[1]{(\textrm{#1})}
%\newcommand{\bracket}[1]{\left(#1\right)}
%\newcommand{\dbar}[1]{\Vert#1\Vert}
%\newcommand{\one}[1]{\mathbbm{1}\{#1\}}

%\def\bm{\boldmath}
%\def\defeq{\stackrel{\textrm{\tiny{def}}}{=}}
\def\mit{\mathit}
\def\defeq{:=}
\def\eps{\epsilon}
\def\fr{\frac}
\def\-{\mbox{-}}
\def\ol{\overline}
\def\real{\mathbb{R}}
\def\intdom{\mathbb{N}}

\def\tO{\tilde{O}}
\def\tOmega{\tilde{\Omega}}

\def\lc{\left \lceil}
\def\lf{\left \lfloor}
\def\rc{\right \rceil}
\def\rf{\right \rfloor}
\newcommand{\ceil}[1]{\lceil #1 \rceil}
\newcommand{\floor}[1]{\lfloor #1 \rfloor}

\def\nn{\nonumber}

\def\Pr{\mathbf{Pr}}
%\def\expt{\mathbf{E}}
\def\var{\mathbf{Var}}

\def\dcl{\{\!\!\{}
\def\dcr{\}\!\!\}}
\def\bigdcl{\Big\{\!\!\Big\{}
\def\bigdcr{\Big\}\!\!\Big\}}
\def\bigmid{\textrm{ $\Big|$ }}


\DeclareMathOperator*{\argmin}{arg\,min}
\DeclareMathOperator*{\argmax}{arg\,max}
\DeclareMathOperator*{\polylog}{polylog}
\DeclareMathOperator*{\poly}{poly}
\DeclareMathOperator*\expt{\mathbf{E}}
%\DeclareMathOperator*\Pr{\mathbf{Pr}}

%===== misc =====

\def\done{\qed \vspace{2mm}}	% end of proof
\def\tbc{\hspace*{\fill} $\textrm{{\em (to be continued)}}\blacktriangle$ \vspace{2mm}}
%\def\done{\hspace*{\fill} $\Box$}	% end of proof
\allowdisplaybreaks
%===== coloring =====

\newcommand{\red}[1]{\textcolor{red}{#1}}
\newcommand{\blue}[1]{\textcolor{blue}{\bf #1}}
\newcommand{\purple}[1]{\textcolor{purple}{\bf #1}}
\newcommand{\todo}[1]{\textcolor{red}{\bf [TO DO: #1]}}
