\noindent \underline{\em Proof of Statement (1).} Consider any tuple $\bm{t} = (r_1,...,r_{k-3}, h', v) \in \J(R_1,...,R_{k-3},H',V)$. Let $h$ be the full segment of $h'$. Since $\bm{t} \in \J(R_1,...,R_{k-3},H',V)$, $B_\bm{t}$, which is the intersection point of $h'$ and $v$, is in $r_i$ for $i \in [k-3]$. As $h$ is the full segment of $h'$, $B_\bm{t} \in h$. Since $v$ also contains $B_\bm{t}$, $B_\bm{t}$ is also the intersection of $h$ and $v$. Let $r_{k-2}$ be an arbitrary rectangle in $\contained_{R_{k-2}}(\bm{t})$. By the definition of $\contained_{R_{k-2}}(\bm{t})$, $r_{k-2}$ covers effective segment of $h'$, whose right enpoint is $B_{\bm{t}}$. Therefore, $B_{\bm{t}}$ is in $r_{i}$ for $i \in [k-2]$, and $\bm{t} \in \J(R_1,...,R_{k-2},H,V)$. Moreover, since the left endpoint of the effective segment of $h'$ is the same as the left endpoint of $h$, $r_{k-2}$ contains the left endpoint of $h$. Therefore, $\bm{t} \in \J_2$.

\vgap 

\noindent \underline{\em Proof of Statement (2).} Let $(r_1,...,r_{k-2},h,v)$, $h'$, $\bm{t}$ be as defined in the statement. First, we prove that $\bm{t}\in \J(R_1,...,R_{k-3}, H', V)$.
Let $p$ be the intersection of $h$ and $v$. Since $(r_1,...,r_{k-2},h,v)\in \J_2$, $p$ is in $r_i$ for $i \in [k-2]$.
We argue that $p$ is also the intersection of $h'$ and $v$, which implies that $h'\cap v$ is covered by $r_i$, $i \in [k-3]$ and $\bm{t} \in \J(R_1,...,R_{k-3},H',V)$. Since $p \in v$, it suffices to show that $p \in h'$. Let $h = [x_1,x_2]\times y$ and $p = (x,y)$. Since $h' \subseteq h$, we can denote $h'$ as $[x_1, x_2'] \times y$, where $x_1 \le x_2' \le x_2$. Then, we only need to show that $x \in [x_1, x_2']$. Since $p \in h$, $x \ge x_1$. Since $(r_1,...,r_{k-2},h,v) \in \J_2$, $r_{k-2}$ is a left-end covering rectangle of $h$ in $R_{k-2}$, that also covers the point $p \in h$. By the definition of trimmed segment (see Secton~\ref{sec:hv:type2}), the x-coordinate of $p$ should not exceed $x_2'$. Hence, $x \in [x_1, x_2']$, and $p \in h'$.

\vgap 

Next, we will show that $r_{k-2}\in \contained_{R_{k-2}}(\bm{t})$. As proven above, $B_\bm{t} = h'\cap v = p$. Recall that $p = (x,y)$. We can denote the effective horizontal segment of $\bm{t}$ as $[x_1, x] \times y$. Then, we only need to prove that $[x_1,x] \subseteq [\xleft(r_2), \xright(r_2)]$ and $y \in [\ybot(r_2), \ytop(r_2)]$. Since $r_{k-2}$ is a left-end covering rectangle of $h$, we have  $x_1 \in [\xleft(r_2), \xright(r_2)]$ and $y \in [\ybot(r_2), \ytop(r_2)]$. As we show in the last paragrah, $r_{k-2}$ also covers $p$. Therefore, $x \in [\xleft(r_2), \xright(r_2)]$, and $[x_1,x] \subseteq [\xleft(r_2), \xright(r_2)]$. Hence, $r_{k-2}\in \contained_{R_{k-2}}(\bm{t})$.

\vgap 

\noindent \underline{\em Proof of Statement (3).}
 This statement follows from Statement (1). More specifically, we can map each pair $(\bm{t}, r)\in \J(R_1,...,R_{k-3},H',V) \times \contained_{R_{k-2}}(\bm{t})$ to the $k$-tuple $(\bm{t}[1], ..., \bm{t}[k-3], r, h, \bm{t}[k-1])$ defined in Statement (1). By Statement (1), this $k$-tuple belongs to $\J_2$. Then, it remains to prove that no two distinct pair $(\bm{t}_1, r_1)$ and $(\bm{t}_2, r_2)$ will be mapped to the same tuple in $\J_2$. Assume to the contrary, $(\bm{t}_1, r_1)$ and $(\bm{t}_2, r_2)$ are mapped to the same tuple $\bm{t} \in \J_2$. Then, $r_1 = \bm{t}[k-2] = r_2$, and $\bm{t}_1[i] = \bm{t}[i] = \bm{t}_2[i]$ for $i \in [k-3]$. Moreover, both $\bm{t}_1[k-2]$ and $\bm{t}_2[k-2]$ are the trimmed segment of $\bm{t}[k-1]$. Since the trimmed segment is unique, $\bm{t}_1[k-2] = \bm{t}_2[k-2]$. Hence, $\bm{t}_1 = \bm{t}_2$, which contradicts the assumption. Therefore, we have $|\J(R_1,...,R_{k-3},H',V)| \le |\J_2| \le \out$, where the second inequality is due to Definition \eqref{eqn:hv:type2:J2}.
