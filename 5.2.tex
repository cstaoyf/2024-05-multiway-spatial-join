\noindent \underline{\em Proof of Statement (1).} Consider any tuple $\bm{t} = (r_1,...,r_{k-3}, h', v) \in \J(R_1,...,R_{k-3},H',V)$. Let $h$ be the full segment of $h'$. Following our conventions, $B_\bm{t}$ denotes the intersection point of $h'$ and $v$, and it is covered by the rectangles $r_1,...,r_{k-3}$. As $h$ is the full segment of $h'$, we have $B_\bm{t} \in h$, and $B_\bm{t}$ is also the intersection of $h$ and $v$. Let $r_{k-2}$ be an arbitrary rectangle in $\contained_{R_{k-2}}(\bm{t})$. By the definition of $\contained_{R_{k-2}}(\bm{t})$, $r_{k-2}$ covers the effective horizontal segment of $h'$, whose right enpoint is $B_{\bm{t}}$. Therefore, $B_{\bm{t}}$ is covered by each of the rectangles $r_1,...,r_{k-2}$. Equivalently, $h \cap v \bigcap_{i = 1}^{k-2} r_i = B_\bm{t} \neq \emptyset$. Hence,
$\bm{t} \in \J(R_1,...,R_{k-2},H,V)$. To show that $\bm{t} \in \J_2$, we only need to prove that $r_{k-2}$ covers the left endpoint of $h$. It is true because $r_{k-2}$ covers the effective horizontal segment of $h'$, whose left endpoint is the same as the left endpoint of $h$. 

\vgap 

\noindent \underline{\em Proof of Statement (2).} Let $(r_1,...,r_{k-2},h,v)$, $h'$, $\bm{t}$ be as defined in the statement. First, we prove that $\bm{t}\in \J(R_1,...,R_{k-3}, H', V)$. It suffices to show that the intersection of $h'$ and $v$ is contained in $r_1,...,r_{k-3}$. We will prove this by showing that $h'\cap v$ is the same as $h\cap v$, which is covered by $r_1,...,r_{k-2}$, because $\bm{t} \in \J(R_1,...,R_{k-2},H,V)$.

\vgap 

Let $p$ be the intersection of $h$ and $v$. Since $h' \subseteq h$, the intersection of $h'$ and $v$ is either empty or the same as $p$.
Therefore, to prove that $h' \cap v = p$, we only need to show that $p \in h'$. Let $h = [x_1,x_2]\times y$ and $p = (x,y)$. Since $h' \subseteq h$, we can denote $h'$ as $[x_1, x_2'] \times y$, where $x_1 \le x_2' \le x_2$. Then, $p \in h'$ is equivalent to $x \in [x_1, x_2']$. Since $p \in h$, we have $x \ge x_1$. By the definition of trimmed segmemt, $(x_2',y)$ is the rightmost point on $h$ satisfying that a left-end covering rectangle of $h$ in $R_{k-2}$ covers it. As $(r_1,...,r_{k-2},h,v) \in \J_2$, the point $p \in h$ is covered by the rectangle $r_{k-2}$, which is a left-end covering rectangle of $h$. Therefore, the x-coordinate of $p$ should not exceed $x_2'$. As a result, we have $x \in [x_1, x_2']$, and $p \in h'$.

\vgap 

Next, we aim at proving $r_{k-2}\in \contained_{R_{k-2}}(\bm{t})$. As proven above, $h'\cap v= p$. In another word, $B_\bm{t} = p$. Recall that we defind $p = (x,y)$, and $h' = [x_1,x_2']\times y$. Thus, we can denote the effective horizontal segment of $\bm{t}$ as $[x_1, x] \times y$. For our goal, we only need to show that $[x_1,x] \subseteq [\xleft(r_2), \xright(r_2)]$ and $y \in [\ybot(r_2), \ytop(r_2)]$. 
\begin{itemize}
    \item Since $r_{k-2}$ is a left-end covering rectangle of $h$, we have  $x_1 \in [\xleft(r_2), \xright(r_2)]$. As we show in the last paragrah, $r_{k-2}$ also covers $p$, whose x-coordinate, denoted as $x$, satisfies $x_1 \le x$. Therefore, $x \in [\xleft(r_2), \xright(r_2)]$, and $[x_1,x] \subseteq [\xleft(r_2), \xright(r_2)]$.
    \item The y-coordinate of $h$ is $y$. Since $r_{k-2}$ intersects $h$, $y \in [\ybot(r_2), \ytop(r_2)]$.
\end{itemize}

\vgap 

\noindent \underline{\em Proof of Statement (3).}
We will provide a way to map each pair $(\bm{t'}, r)$ satisfyint that $\bm{t'}\in \J(R_1,...,R_{k-3},H',V)$ and $r \in \contained_{R_{k-2}}(\bm{t})$ to a unique tuple in $\J_2$. Consider an arbitrary such pair $(\bm{t'}, r)$. Let $h$ be the full segment of $\bm{t'}[k-2]$. We map $(\bm{t'}, r)$ to the tuple $\bm{t} = (\bm{t}'[1],...,\bm{t}'[k-3], r, h, \bm{t}'[k-1])$. By Statement (1) and the fact that $r\in \contained_{R_{k-2}}(\bm{t'})$, the $k$-tuple $\bm{t}$ belongs to $\J_2$. 

\vgap 

It remains to prove that no two distinct pair $(\bm{t}_1, r_1)$ and $(\bm{t}_2, r_2)$ will be mapped to the same tuple in $\J_2$. Assume, on the contrary, $(\bm{t}_1, r_1)$ and $(\bm{t}_2, r_2)$ are mapped to the same tuple $\bm{t} \in \J_2$. Then, $r_1 = \bm{t}[k-2] = r_2$, and $\bm{t}_1[i] = \bm{t}[i] = \bm{t}_2[i]$ for $i \in [k-3]$. Moreover, both $\bm{t}_1[k-2]$ and $\bm{t}_2[k-2]$ are the trimmed segment of $\bm{t}[k-1]$. Since the trimmed segment is unique, $\bm{t}_1[k-2] = \bm{t}_2[k-2]$. Hence, $\bm{t}_1 = \bm{t}_2$, which contradicts the assumption. Therefore, we have $|\J(R_1,...,R_{k-3},H',V)| \le |\J_2| \le \out$, where the second inequality is due to Definition \eqref{eqn:hv:type2:J2}.
