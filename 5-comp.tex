W.l.o.g., we assume that each segment in the input $H$ is given a distinct integer ID in $[|H|]$. This allows us to create an array of size $|H|$ and allocate an array cell to each $h \in H$, with the property that the cell can be accessed by the ID of $h$ in constant time.

\vgap

To compute $H^*$, we start by deriving $\minleft(h')$ for each segment $h'$ in $H'$. For this purpose, first initialize $\minleft(h') = \infty$ for each such $h'$. Recall that $h'$ is the trimmed segment of some segment $h$ in $H$. We store $\minleft(h')$ in the array cell allocated to $h$.

\vgap

Then, we scan $\J(R_1,...,R_{k-3},H',V)$. For each tuple $\bm{t}$ therein, update in constant time $\minleft(h')$ to the minimum between its current value and the x-coordinate of $\bm{t}[k-1]$. The scan requires $O(n + \out)$ time.

\vgap

Finally, we construct $H^*$ by collecting the minimal segment (see definition in \eqref{eqn:hv:type2:H*}) of every segment $h' \in H'$ with $\minleft(h') \neq \infty$. This step takes $O(|H'|) = O(n)$ time.
