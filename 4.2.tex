First, we prove (1). By the definition of $\dcross_H(\bm{t})$ and $\dcross_V(\bm{t})$, $h$ and $v$ cross $B_\bm{t}$. Thus, the intersection of $h$ and $v$ is a point in $B_\bm{t}$. Since $B_\bm{t}$ is the intersection of $\bm{t}[1],...,\bm{t}[k-2]$, the intersection of $h$ and $v$ lies in all the rectangles $r_1,...,r_{k-2}$. Therefore, $h\cap v\bigcap_{i = 1}^{k-2}r_i \neq \emptyset$, and  the tuple $(r_1,...,r_{k-2},h,v)$ is a result tuple in $\J(R_1,...,R_{k-2},H,V)$. 

Next, we prove (2). Since $(r_1,...,r_{k-2},h,v)\in \J_1$, both $h$
 and $v$ cross the rectangles $r_1,...,r_{k-2}$ which means that for $i\in[k-2]$, the y-coorninate of $h$ is in the range $[\ybot(r_i'), \ytop(r_i')]$ and the x-coordinate of $v$ is in $[\xleft(r_i'), \xright(r_i')]$. Additionally, for each $i\in[k-2]$, the x-range (resp. y-range) of $r_i'$ is a subset of the x-range (resp. y-range) of $r_i$, so $h$ (resp. $v$) crosses the rectangles $r_i'$. Hence, the point $h \cap v$ is in $B_\bm{t} = \bigcap_{i = 1}^{k-2}r_i'$, and $h, v$ cross $B_{\bm{t}}$. The former implies that $\bm{t} \in \J(R_1',...,R_{k-2}')$, and the latter implies that $h \in \dcross_H(\bm{t})$ and $v \in \dcross_V(\bm{t})$.

Finally, we prove (3). 
For each $\bm{t'}\in  \J(R_1',...,R_{k-2}')$ and each $h \in \dcross_H(\bm{t'})$, we will identify a unique tuple $\bm{t}$ in $\J(R_1,...,R_{k-2},H,V)$ defined as follows. Let $r_i$ be the full rectangle of $r_i'$ for $i \in [k]$, and $v$ be the segment $v_\vdash$ of $\gleft(\bm{t'})$. Let $\bm{t} = (r_1,...,r_{k-2},h,v)$. Since $h$ crosses $B_{\bm{t'}}$ by the definition of $\dcross_H(\bm{t'})$, the $h\cap v$ is a point in $B_{\bm{t'}}$. 
Therefore, $h\cap v \in B_{\bm{t'}} \subseteq B_{\bm{t}}$, and $\bm{t}\in \J(R_1,...,R_{k-2},H,V)$. he result tuple is unique because for any two distinct tuples $\bm{t'_1}$ and $\bm{t'_2}$ in $\J(R_1',...,R_{k-2}')$, there exists an integer $i \in [k-2]$ s.t. the full rectangles of $\bm{t'_1}[i]$ and $\bm{t'_2}[i]$ are different, which means the $i$-th components of their associating tuples $\bm{t_1}, \bm{t_2} \in \J(R_1,...,R_{k-2},H,V)$ are different. As a result, $\sum_{\bm{t}\in \J(R_1',...,R_{k-2}')}  |\dcross_H(\bm{t})| \leq \out$. 

Similarly, for each $\bm{t'}\in  \J(R_1',...,R_{k-2}')$ and each $v \in \dcross_V(\bm{t'})$, we can identify a unique tuple $\bm{t}$ in $\J(R_1,...,R_{k-2},H,V)$ defined as follows. Let $r_i$ be the full rectangle of $r_i'$ for $i \in [k]$, and $h$ be the segment $h_\bot$ of $\gbot(\bm{t'})$. Let $\bm{t} = (r_1,...,r_{k-2},h,v)$. Since $v$ crosses $B_{\bm{t'}}$, $h\cap v \in B_{\bm{t'}} \subseteq B_{\bm{t}}$, and $\bm{t}\in \J(R_1,...,R_{k-2},H,V)$. As a result, $\sum_{\bm{t}\in \J(R_1',...,R_{k-2}')}  |\dcross_V(\bm{t})| \leq \out$. 