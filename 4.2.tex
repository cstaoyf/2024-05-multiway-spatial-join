



\vgap 


\noindent \underline{\em Proof of Statement (3).}  First, we prove that $\sum_{\bm{t}\in \J(R_1',...,R_{k-2}')}  |\dcross_H(\bm{t})|$ does not exceeds $\out$. Then, we can apply similar arguments to show that $\sum_{\bm{t}\in \J(R_1',...,R_{k-2}')}  |\dcross_V(\bm{t})| \leq |\out|$. 
 
\vgap 

We will provide a way to map each pair $(\bm{t},h)$ in $\J(R_1',...,R_{k-2}')\times \dcross_H(\bm{t})$ to a unique tuple in the H-V $k$-SJ result $\J(R_1,...,R_{k-2},H,V)$.

\vgap 

To explain the mapping, consider any pair $(\bm{t},h) \in \J(R_1',...,R_{k-2}')\times \dcross_H(\bm{t})$. Recall that $B_\bm{t}$ is the intersection of the $k-2$ rectangles $\bm{t}[1], ..., \bm{t}[k-2]$. Since $h \in \dcross_H(\bm{t})$ (see definition~\eqref{eqn:dcross-H}), $h$ crosses $B_\bm{t}$. Let $p$ be the intersection point of $h$ and the left edge of $B_\bm{t}$. Define $i$ to be the value in $[k-2]$ such that $\bm{t}[i] = \gleft(\bm{t})$. 

\vgap

Recall that each $\bm{t}[j]$ (for $j \in [k-2]$) is a trimmed rectangle; let us denote by $r_j$ the full rectangle of $\bm{t}[j]$. Regarding the value $i$ decided earlier, let $v$ be the leftmost segment in $V$ that crosses $r_i$. By the way trimmed rectangles are computed, we know that $v$ must contain the left edge of rectangle $\bm{t}[i]$. This means that point $p$ must be the intersection point of $h$ and $v$.

\vgap 

We argue that the $k$-tuple $(r_1, ..., r_{k-2}, h, v)\in \J(R_1,..., R_{k-2},H,V)$. Clearly, $r_j$ covers $\bm{t}[j]$ for all $j \in [k-2]$. Thus, $\bigcap_{j=1}^{k-2} r_j$ covers $\bigcap_{j=1}^{k-2} \bm{t}[j]$, which is $B_\bm{t}$. Hence, point $p$ --- which is the intersection point of $h$ and the left edge of $B_\bm{t}$ --- falls in $\bigcap_{j=1}^{k-2} r_j$, indicating $h \cap v \cap \bigcap_{j=1}^{k-2} r_j \ne \emptyset$. We map $(\bm{t},h)$ to $(r_1, ..., r_{k-2}, h, v)$.

\vgap

It remains to prove that no two distinct pairs $(\bm{t}_1, h_1), (\bm{t}_2, h_2) \in \J(R_1',...,R_{k-2}')\times \dcross_H(\bm{t})$ can be mapped to the same tuple in $\J(R_1,..., R_{k-2},H,V)$. Assume that this is not true, namely, both $(\bm{t}_1, h_1)$ and $(\bm{t}_2, h_2)$ are mapped to the tuple $(r_1, ..., r_{k-2}, h, v) \in \J(R_1,..., R_{k-2}, H, V)$. However, in this case, $\bm{t}_1[j]$ is the trimmed rectangle of $r_j$, and $\bm{t}[k-1] = h_1$. At the same time, $\bm{t}_2[j]$ is the trimmed rectangle of $r_j$, and $\bm{t}[k-1] = h_2$. We thus have $\bm{t}_1 = \bm{t}_2$ and $h_1 = h_2$, giving a contradiction. Therefore, we have $\sum_{\bm{t}\in \J(R_1',...,R_{k-2}')}  |\dcross_H(\bm{t})| \leq |\out|$.

\vgap 

Analogously, we can map each pair $(\bm{t},v)$ in $\J(R_1',...,R_{k-2}')\times \dcross_V(\bm{t})$ to a unique tuple $(r_1,...,r_{k-2},h,v)$, where 
\begin{itemize}
    \item $r_j$ is the full rectangle of $\bm{t}[j]$ for all $j \in [k-2]$,
    \item $h$ is the lowest segment in $H$ that crosses $r_i$, where $i$ is the value in $[k-2]$ such that $\bm{t}[i] = \gbot(\bm{t})$.
\end{itemize}
Let $p$ be the intersection of $v$ and the bottom edge of $B_\bm{t}$. We argue that the $k$-tuple $(r_1, ..., r_{k-2}, h, v)\in \J(R_1,..., R_{k-2},H,V)$. By the way trimmed rectangles are computed, we know that $h$ must contain the bottom edge of rectangle $\bm{t}[i]$, which is a superset of the bottom edge of $B_\bm{t}$. This means that point $p$ must be the intersection point of $h$ and $v$. As shown before, $r_j$ covers $B_\bm{t}$. Hence, point $p$ falls in $\bigcap_{j=1}^{k-2} r_j$, indicating $h \cap v \cap \bigcap_{j=1}^{k-2} r_j \ne \emptyset$. 

\vgap 

It suffices to show that no two distinct pairs $(\bm{t}_1, v_1), (\bm{t}_2, v_2) \in \J(R_1',...,R_{k-2}')\times \dcross_V(\bm{t})$ can be mapped to the same tuple in $\J(R_1,..., R_{k-2},H,V)$. Assume the contrary, namely, both $(\bm{t}_1, v_1)$ and $(\bm{t}_2, v_2)$ are mapped to the tuple $(r_1, ..., r_{k-2}, h, v) \in \J(R_1,..., R_{k-2}, H, V)$. However, in this case, $\bm{t}_1[j]$ is the trimmed rectangle of $r_j$, and $\bm{t}[k-2] = v_1$. At the same time, $\bm{t}_2[j]$ is the trimmed rectangle of $r_j$, and $\bm{t}[k-2] = v_2$. We thus have $\bm{t}_1 = \bm{t}_2$ and $v_1 = v_2$, giving a contradiction.
Therefore, we have $\sum_{\bm{t}\in \J(R_1',...,R_{k-2}')}  |\dcross_V(\bm{t})| \leq |\out|$.
