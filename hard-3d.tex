The input to the 3-SJ problem in 3D space comprises
 three sets of axis-parallel boxes $S_1$, $S_2$, and $S_3$. 
% A box $s_i \in S_i$, for $i \in [3]$, is of the form $s_i = [x_1,x_2]\times[y_1,y_2]\times[z_1,z_2]$. 
The goal is to find the set of tuples $(s_1,s_2,s_3)$ such that $s_i \in S_i$ for $i \in [3]$ and $s_1 \cap s_2 \cap s_3 \neq \emptyset$. Denote the output set as $\J(S_1,S_2,S_3)$.
Let $n:=|S_1|+|S_2|+|S_3|$ be the input size, and let $\out = |\J(S_1,S_2,S_3)|$.

The triangle detection problem is defined as follows. Given a graph $G = (V,E)$, where $m:=|E|$, we want to find out whether there exists a triangle in the graph. We will reduce this problem to an instance of the 3-SJ problem in 3D space with input size $12m=O(m)$. First, we represent each node in $G$ as a unique integer in $[|V|]$, and create three empty sets of 3D boxes $S_1$, $S_2$, $S_3$. For each edge $\set{u,v}\in E$, we add the following boxes to the sets $S_1$:
\begin{itemize}
    \item $[u,u]\times [v,v]\times(-\infty,u-1]$,
    \item $[u,u]\times [v,v]\times[u+1,+\infty)$,
    \item $[v,v]\times [u,u] \times (-\infty,v-1]$,
    \item $[v,v]\times [u,u] \times [v+1,+\infty)$;
\end{itemize}
and the following boxes to the sets $S_2$:
\begin{itemize}
    \item $(-\infty,u-1]\times[u,u]\times [v,v]$,
    \item $[u+1,+\infty)\times[u,u]\times [v,v]$,
    \item $(-\infty,v-1] \times[v,v]\times [u,u]$,
    \item $ [v+1,+\infty)\times [v,v]\times [u,u]$;
\end{itemize}
and the following boxes to the sets $S_3$:
\begin{itemize}
    \item $[u,u]\times[v+1,+\infty)\times [v,v]$,
    \item $[u,u]\times(-\infty,v-1]\times [v,v]$,
    \item $[v,v]\times[u+1,+\infty)\times [u,u]$,
    \item $[v,v]\times(-\infty,u-1]\times [u,u]$.
\end{itemize}
Now we prove that there is a triangle in $G$ iff.
 $\J(S_1,S_2,S_3) \neq \emptyset$. If there is a triangle in $G$ with nodes $u$, $v$, $w$ satisfying that $u < v < w$, then the three boxes:
\begin{itemize}
    \item $[u,u]\times [v,v]\times[u+1, +\infty)\in S_1$,
    \item $(-\infty,v-1]\times[v,v]\times [w,w] \in S_2$,
    \item $[u,u]\times(-\infty, w-1]\times [w,w] \in S_3$,
\end{itemize}
will form a result tuple in $\J(S_1,S_2,S_3)$. Conversely, assume the existence of a tuple $(s_1,s_2,s_3)\in \J(S_1,S_2,S_3)$, s.t. $s_1 = [v,v]\times [u,u]\times[v+1, +\infty)$, $s_2 = (-\infty,u-1]\times[u,u]\times [w,w]$, and $s_3 = [v,v]\times(-\infty, w-1]\times [w,w]$. We can reason about the other forms of result tuples analogously. From the reduction, we have $v< u< w$, and the edges $\set{u,v}$, $\set{v,w}$, and $\set{u,w}$ form a triangle in $G$. 
Hence, if we can detect whether there is an output for the 3-SJ problem in 3D space in $O(n\polylog n)$ time, then we can solve the triangle detection problem 
in $O(m \polylog m)$ time. Furthermore, it is standard to convert an $O(n \log n + \out)$-time reporting algorithm for the 3-SJ problem in 3D space to an algorithm that detects whether there is an output for the same problem in $O(n \polylog n)$ time. More specifically, when $\out = 0$, the reporting algorithm runs in $cn\polylog n$ time for some constant $c$. We can simply run this algorithm for $cn\polylog n$ time and check the output and whether it terminates. If the algorithm terminates and reports no output, then we declare that $\out = 0$. Otherwise, the algorithm either reports an output or runs for more than $cn\polylog n$ time. Both cases indicate that there is at least one output for the 3-SJ problem in 3D space.  The input to the 3-SJ problem in 3D space comprises
 three sets of axis-parallel boxes $S_1$, $S_2$, and $S_3$. 
% A box $s_i \in S_i$, for $i \in [3]$, is of the form $s_i = [x_1,x_2]\times[y_1,y_2]\times[z_1,z_2]$. 
The goal is to find the set of tuples $(s_1,s_2,s_3)$ such that $s_i \in S_i$ for $i \in [3]$ and $s_1 \cap s_2 \cap s_3 \neq \emptyset$. Denote the output set as $\J(S_1,S_2,S_3)$.
Let $n:=|S_1|+|S_2|+|S_3|$ be the input size, and let $\out = |\J(S_1,S_2,S_3)|$.

The triangle detection problem is defined as follows. Given a graph $G = (V,E)$, where $m:=|E|$, we want to find out whether there exists a triangle in the graph. We will reduce this problem to an instance of the 3-SJ problem in 3D space with input size $12m=O(m)$. First, we represent each node in $G$ as a unique integer in $[|V|]$, and create three empty sets of 3D boxes $S_1$, $S_2$, $S_3$. For each edge $\set{u,v}\in E$, we add the following boxes to the sets $S_1$:
\begin{itemize}
    \item $[u,u]\times [v,v]\times(-\infty,u-1]$,
    \item $[u,u]\times [v,v]\times[u+1,+\infty)$,
    \item $[v,v]\times [u,u] \times (-\infty,v-1]$,
    \item $[v,v]\times [u,u] \times [v+1,+\infty)$;
\end{itemize}
and the following boxes to the sets $S_2$:
\begin{itemize}
    \item $(-\infty,u-1]\times[u,u]\times [v,v]$,
    \item $[u+1,+\infty)\times[u,u]\times [v,v]$,
    \item $(-\infty,v-1] \times[v,v]\times [u,u]$,
    \item $ [v+1,+\infty)\times [v,v]\times [u,u]$;
\end{itemize}
and the following boxes to the sets $S_3$:
\begin{itemize}
    \item $[u,u]\times[v+1,+\infty)\times [v,v]$,
    \item $[u,u]\times(-\infty,v-1]\times [v,v]$,
    \item $[v,v]\times[u+1,+\infty)\times [u,u]$,
    \item $[v,v]\times(-\infty,u-1]\times [u,u]$.
\end{itemize}
Now we prove that there is a triangle in $G$ iff.
 $\J(S_1,S_2,S_3) \neq \emptyset$. If there is a triangle in $G$ with nodes $u$, $v$, $w$ satisfying that $u < v < w$, then the three boxes:
\begin{itemize}
    \item $[u,u]\times [v,v]\times[u+1, +\infty)\in S_1$,
    \item $(-\infty,v-1]\times[v,v]\times [w,w] \in S_2$,
    \item $[u,u]\times(-\infty, w-1]\times [w,w] \in S_3$,
\end{itemize}
will form a result tuple in $\J(S_1,S_2,S_3)$. Conversely, assume the existence of a tuple $(s_1,s_2,s_3)\in \J(S_1,S_2,S_3)$, s.t. $s_1 = [v,v]\times [u,u]\times[v+1, +\infty)$, $s_2 = (-\infty,u-1]\times[u,u]\times [w,w]$, and $s_3 = [v,v]\times(-\infty, w-1]\times [w,w]$. We can reason about the other forms of result tuples analogously. From the reduction, we have $v< u< w$, and the edges $\set{u,v}$, $\set{v,w}$, and $\set{u,w}$ form a triangle in $G$. 
Hence, if we can detect whether there is an output for the 3-SJ problem in 3D space in $O(n\polylog n)$ time, then we can solve the triangle detection problem 
in $O(m \polylog m)$ time. Furthermore, it is standard to convert an $O(n \log n + \out)$-time reporting algorithm for the 3-SJ problem in 3D space to an algorithm that detects whether there is an output for the same problem in $O(n \polylog n)$ time. More specifically, when $\out = 0$, the reporting algorithm runs in $cn\polylog n$ time for some constant $c$. We can simply run this algorithm for $cn\polylog n$ time and check the output and whether it terminates. If the algorithm terminates and reports nothing, then we declare that there is no output. Otherwise, the algorithm either reports an output or runs for more than $cn\polylog n$ time. Both cases indicate that there is at least one output for the 3-SJ problem in 3D space.  