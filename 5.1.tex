We prove by mapping each tuple $\J(R_1,...,R_{k-3},H',V)$ to a unique tuple in $\J_2$. 
To define the mapping, consider an arbitrary tuple $(r_1,...,r_{k-3}, h', v) \in \J(R_1,...,R_{k-3},H',V)$. Let $h$ be the full segment of $h'$. By the definition of $h'$, there exists a rectangle $r_{k-2}\in R_{k-2}$ that covers the left endpoint of $h$ and the right endpoint of $h'$. In another word, $r_{k-2}$ covers $h'$. Let $\bm{t} = (r_1,...,r_{k-3}, r_{k-2},h,v)$. We map the tuple $(r_1,...,r_{k-3}, h', v)$ to $\bm{t}$. 

\vgap 

We argue that $\bm{t}\in \J_2$. Let $p$ be the intersection of $h'$ and $v$. Since $(r_1,...,r_{k-3}, h', v) \in \J(R_1,...,R_{k-3},H',V)$, $p \in r_i$ for $i \in [k-3]$. As we proved in the last paragraph, $r_{k-2}$ covers $h'$, so $p \in r_{k-2}$. Additionally, $r_{k-2}$ covers the left endpoint of $h$. Therefore, $\bm{t} \in \J_2$.

\vgap

It remains to prove that no two distinct tuple $\bm{t}_1, \bm{t}_2$ will be mapped to the same tuple in $\J_2$. Assume to the contrary, $\bm{t}_1, \bm{t}_2$ are mapped to the same tuple $\bm{t} \in \J_2$. Then, $\bm{t}_1[i] = \bm{t}[i] = \bm{t}_2[i]$ for $i\in [k-3]$ and $i = k-1$. Moreover, $\bm{t}_1[k-2]$ and $\bm{t}_2[k-2]$ are both trimmed segment of $\bm{t}[k-2]$. Since the trimmed segment is unique, $\bm{t}_1[k-2] = \bm{t}_2[k-2]$. Hence, $\bm{t}_1 = \bm{t}_2$, which contradicts the assumption.  

\vgap

Therefore, we have $|\J(R_1,...,R_{k-3},H',V)| \le |\J_2| \le \out$, where the second inequality is due to Definition \eqref{eqn:hv:type2:J2}. 