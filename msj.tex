\documentclass[sigconf]{acmart}
% \documentclass{...}
%\input{./def/yf-formatting}
\input{./def/yf-def}

%=================================
%Yufei's stuff
%\usepackage{amsmath}
\usepackage{balance}
%\usepackage{times}
\usepackage{microtype}

\def\vgap{\vspace{0mm}}
\def\extraspacing{\vspace{1mm} \noindent}
\def\figcapup{\vspace{-2mm}}
\def\figcapdown{\vspace{-4mm}}

\def\A{\mathcal{A}}
\def\B{\mathcal{B}}
\def\C{\mathcal{C}}
\def\E{\mathcal{E}}
\def\G{\mathcal{G}}
\def\I{\mathcal{I}}
\def\II{\mathscr{I}}
\def\L{\mathcal{L}}
\def\Q{\mathcal{Q}}
\def\R{\mathcal{R}}
\def\T{\mathcal{T}}
\def\U{\mathcal{U}}
\def\V{\mathcal{V}}
\def\X{\mathcal{X}}
\def\XX{\mathscr{X}}
\def\Y{\mathcal{Y}}
\def\YY{\mathscr{Y}}
\def\Z{\mathcal{Z}}

\def\out{\mathrm{OUT}}

\allowdisplaybreaks
%=================================

%====== from acm ======
\acmDOI{}
\acmISBN{}

\acmConference[]{...}{...}{...}
\acmYear{...}
\copyrightyear{}
\acmArticle{}
\acmPrice{}
%======================



\begin{document}
%\begin{sloppy}
    
\title{Optimal Multiway Spatial Joins}

% \author{}
% \affiliation{
% 	\institution{Chinese University of Hong Kong}
% 	\city{Hong Kong}
% 	\country{China}}	

\author{}


\begin{abstract}
    In the {\em multiway spatial join} problem, we have a constant number $k \ge 2$ sets of axis-parallel rectangles $R_1, R_2, ..., R_k$ in 2D space and need to report all $(r_1, r_2, ..., r_k) \in R_1 \times R_2 \times ... \times R_k$ such that the $k$ rectangles $r_1, r_2, ..., r_k$ have a non-empty intersection, i.e., $r_1 \cap r_2 \cap ... \cap r_k \neq \emptyset$. The problem, which plays a fundamental role in spatial databases, has been extensively studied for over two decades. In this paper, we show how to settle the problem --- regardless of the constant $k$ --- in $O(n \log n + \out)$ time, where $n = \sum_{i=1}^k |R_i|$ and $\out$ is the total number of $k$-tuples in the result. This significantly improves the state of the art, which is an algorithm with running time $n \log^{\Theta(k)} n + O(\out)$. Our algorithm is comparison-based
\end{abstract}

\maketitle 

\section{Introduction} \label{sec:intro}

\bibliographystyle{plainurl}% the mandatory bibstyle
\bibliography{ref}

\balance

%\end{sloppy}
\end{document}

