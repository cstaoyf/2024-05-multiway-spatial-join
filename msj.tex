\documentclass[sigconf]{acmart}
% \documentclass{...}
%\documentclass[11pt, letterpaper]{article}
\usepackage[margin=1in]{geometry}

% \documentclass[11pt, letterpaper]{article}
% 
% % ----- margins -----
% 
% \topmargin -1.5cm         % read Lamport p.163
% \oddsidemargin -0.04cm    % read Lamport p.163
% \evensidemargin -0.04cm   % same as oddsidemargin but for left-hand pages
% 
% % ----- texts -----
% 
% \textwidth 16.59cm
% \textheight 21.94cm
% 
% % ----- indendts and spacing -----
% 
% \parskip 0pt            	% spacing between paragraphs
% %\renewcommand{\baselinestretch}{1.5}	% uncomment for 1.5 spacing
% 
% \parindent 7mm		      % leading space for paragraphs between lines
% 
% % ----- page # -----
% 
% %\pagestyle{empty}         % uncomment if don't want page numbers





%=== use the these packages if they are not already in use ===
%\usepackage{amsfonts}
%\usepackage{amsmath}
%\usepackage{amssymb}
\usepackage{amsthm}
\usepackage{bbm} 
%\usepackage{cite}
\usepackage{color}
%\usepackage{euscript}
\usepackage{graphicx}
\usepackage{mathrsfs} 
%\usepackage{microtype}
% \usepackage[normalem]{ulem}
% \usepackage{wrapfig}
%=============================================================

%===== control =====
\allowdisplaybreaks

%===== fonts =====
\def\ttt{\texttt}
\def\tsc{\textsc}

%===== spacing =====

\def\extraspacing{\vspace{3mm} \noindent}
\def\figcapup{\vspace{-1mm}}
\def\figcapdown{\vspace{-0mm}}
\def\hgap{\textrm{\hspace{1mm}}}
\def\thmvgap{\vspace{0mm}}
\def\vgap{\vspace{1mm}}


%===== tabbing =====

\def\tab{\hspace{3mm}}
\def\tabpos{\hspace{4mm} \= \hspace{4mm} \= \hspace{4mm} \= \hspace{4mm} \= \hspace{4mm} \= \hspace{4mm} \= \hspace{4mm} \= \hspace{4mm} \= \hspace{4mm} \= \hspace{4mm} \= \hspace{4mm} \= \hspace{4mm} \= \hspace{4mm} \= \hspace{4mm} %\= 
%\= \hspace{4mm} \= \hspace{4mm} \= \hspace{4mm} \= \hspace{4mm} \= \hspace{4mm} \= \hspace{4mm}
\kill}
\newcommand{\mytab}[1]{\begin{tabbing}\tabpos #1\end{tabbing}}

%===== blocks =====

%%%%%%%%%%%%%%%%%%%%%%%%%%%%%%%%
% THEOREMS
%%%%%%%%%%%%%%%%%%%%%%%%%%%%%%%%
% \theoremstyle{plain}
% \newtheorem{theorem}{Theorem}[section]
% \newtheorem{proposition}[theorem]{Proposition}
% \newtheorem{lemma}[theorem]{Lemma}
% \newtheorem{corollary}[theorem]{Corollary}
% \theoremstyle{definition}
% \newtheorem{definition}[theorem]{Definition}
% \newtheorem{assumption}[theorem]{Assumption}
% \theoremstyle{remark}
% \newtheorem{remark}[theorem]{Remark}

% \newtheorem{theorem}{Theorem}
% \newtheorem{lemma}[theorem]{Lemma}
% \newtheorem{corollary}[theorem]{Corollary}
% \newtheorem{proposition}{Proposition}
% \newtheorem{definition}{Definition}
% \newtheorem{problem}{Problem}

\newcommand{\boxminipg}[2]{\begin{center}\fbox{\begin{minipage}{#1}#2\end{minipage}}\end{center}}
\newcommand{\minipg}[2]{\begin{center}\begin{minipage}{#1}#2\end{minipage}\end{center}}
\newcommand{\myitems}[1]{\begin{itemize} #1 \end{itemize}}
\newcommand{\myenums}[1]{\begin{enumerate} #1 \end{enumerate}}
\newcommand{\myfig}[1]{\begin{figure}\centering #1\end{figure}}
\newcommand{\myfigg}[2]{\begin{figure}\centering #1 \figcapup \caption{#2} \figcapdown \end{figure}}
\newcommand{\myfigstar}[2]{\begin{figure*}\centering #1 \figcapup \caption{#2} \figcapdown \end{figure*}}

%===== math macros =====

\newcommand{\bm}[1]{\textrm{\boldmath${#1}$}}
\newcommand{\mb}[1]{\mathbf{#1}}
% \newcommand{\smat}[2]{\left[\begin{tabular}{#1}#2\end{tabular}\right]}
% \newcommand{\bmat}[2]{\left|\begin{tabular}{#1}#2\end{tabular}\right|}
\newcommand{\bmat}[1]{\begin{bmatrix}#1\end{bmatrix}}
\newcommand{\vmat}[1]{\begin{vmatrix}#1\end{vmatrix}}
\newcommand{\myeqn}[1]{\begin{eqnarray}#1\end{eqnarray}}
\newcommand{\myset}[1]{\{#1\}}
\newcommand{\set}[1]{\{#1\}}

\newcommand{\explain}[1]{(\textrm{#1})}
%\newcommand{\bracket}[1]{\left(#1\right)}
%\newcommand{\dbar}[1]{\Vert#1\Vert}
%\newcommand{\one}[1]{\mathbbm{1}\{#1\}}

%\def\bm{\boldmath}
%\def\defeq{\stackrel{\textrm{\tiny{def}}}{=}}
\def\mit{\mathit}
\def\defeq{:=}
\def\eps{\epsilon}
\def\fr{\frac}
\def\-{\mbox{-}}
\def\ol{\overline}
\def\real{\mathbb{R}}
\def\intdom{\mathbb{N}}

\def\tO{\tilde{O}}
\def\tOmega{\tilde{\Omega}}

\def\lc{\left \lceil}
\def\lf{\left \lfloor}
\def\rc{\right \rceil}
\def\rf{\right \rfloor}
\newcommand{\ceil}[1]{\lceil #1 \rceil}
\newcommand{\floor}[1]{\lfloor #1 \rfloor}

\def\nn{\nonumber}

\def\Pr{\mathbf{Pr}}
%\def\expt{\mathbf{E}}
\def\var{\mathbf{Var}}

\def\dcl{\{\!\!\{}
\def\dcr{\}\!\!\}}
\def\bigdcl{\Big\{\!\!\Big\{}
\def\bigdcr{\Big\}\!\!\Big\}}
\def\bigmid{\textrm{ $\Big|$ }}


\DeclareMathOperator*{\argmin}{arg\,min}
\DeclareMathOperator*{\argmax}{arg\,max}
\DeclareMathOperator*{\polylog}{polylog}
\DeclareMathOperator*{\poly}{poly}
\DeclareMathOperator*\expt{\mathbf{E}}
%\DeclareMathOperator*\Pr{\mathbf{Pr}}

%===== misc =====

\def\done{\qed \vspace{2mm}}	% end of proof
\def\tbc{\hspace*{\fill} $\textrm{{\em (to be continued)}}\blacktriangle$ \vspace{2mm}}
%\def\done{\hspace*{\fill} $\Box$}	% end of proof
\allowdisplaybreaks
%===== coloring =====

\newcommand{\red}[1]{\textcolor{red}{#1}}
\newcommand{\blue}[1]{\textcolor{blue}{\bf #1}}
\newcommand{\purple}[1]{\textcolor{purple}{\bf #1}}
\newcommand{\todo}[1]{\textcolor{red}{\bf [TO DO: #1]}}


%=================================
%Yufei's stuff
%\usepackage{amsmath}
\usepackage{balance}
%\usepackage{times}
\usepackage{microtype}

\def\vgap{\vspace{0mm}}
\def\extraspacing{\vspace{1mm} \noindent}
\def\figcapup{\vspace{-2mm}}
\def\figcapdown{\vspace{-4mm}}

\def\A{\mathcal{A}}
\def\B{\mathcal{B}}
\def\C{\mathcal{C}}
\def\E{\mathcal{E}}
\def\G{\mathcal{G}}
\def\I{\mathcal{I}}
\def\J{\mathcal{J}}
\def\II{\mathscr{I}}
\def\L{\mathcal{L}}
\def\P{\mathcal{P}}
\def\Q{\mathcal{Q}}
\def\R{\mathcal{R}}
\def\T{\mathcal{T}}
\def\U{\mathcal{U}}
\def\V{\mathcal{V}}
\def\X{\mathcal{X}}
\def\XX{\mathscr{X}}
\def\Y{\mathcal{Y}}
\def\YY{\mathscr{Y}}
\def\Z{\mathcal{Z}}

\def\out{\mathrm{OUT}}

\allowdisplaybreaks
%=================================

%====== from acm ======
\acmDOI{}
\acmISBN{}

\acmConference[]{...}{...}{...}
\acmYear{...}
\copyrightyear{}
\acmArticle{}
\acmPrice{}
%======================



\begin{document}
%\begin{sloppy}
    
\title{Optimal Multiway Spatial Joins}

% \author{}
% \affiliation{
% 	\institution{Chinese University of Hong Kong}
% 	\city{Hong Kong}
% 	\country{China}}	

\author{}


\begin{abstract}
    In a {\em multiway spatial join}, we are given a constant number $k \geq 3$ of sets containing axis-parallel rectangles in a 2D space, denoted as $R_1, R_2, ..., R_k$. The objective is to identify and report all $k$-tuples $(r_1, r_2, ..., r_k) \in R_1 \times R_2 \times ... \times R_k$ where the rectangles $r_1, r_2, ..., r_k$ have a non-empty intersection, i.e., $r_1 \cap r_2 \cap ... \cap r_k \neq \emptyset$. This problem holds significant importance in spatial databases and has been extensively studied for over two decades. We explain how to settle the problem in $O(n \log n + \out)$ time using $O(n)$ space --- regardless of the constant $k$ --- where $n = \sum_{i=1}^k |R_i|$ and $\out$ is the result size (i.e., the total number of $k$-tuples reported). The time complexity is asymptotically optimal in the class of comparison-based algorithms, to which our solution belongs. Our result significantly improves the state of the art, which is an algorithm with running time $O(n \log^{\Theta(k)} n + \out)$.
\end{abstract}

\maketitle 

\section{Introduction} \label{sec:intro}

This paper revisits the {\em multiway spatial join} (MSJ) problem defined as follows. Let $k$ be a constant integer at least 3. The input of {\em $k$-MSJ} comprises $k$ sets --- denoted as $R_1, R_2, ..., R_k$ --- of axis-parallel rectangles\footnote{A rectangle $r$ in 2D space is {\em axis-parallel} if it has the form $r = [x_1, x_2] \times [y_1, y_2]$.} in $\real^2$. The objective is to find all the $k$-tuples $(r_1, r_2, ..., r_k)$ such that
\myitems{
    \item $r_i \in R_i$ for each $1 \le i \le k$; and
    \item $r_1 \cap r_2 \cap ... \cap r_k \neq \emptyset$, namely, the $k$ rectangles $r_1, r_2, ..., r_k$ have a non-empty intersection.
}
We represent the set of these $k$-tuples as $\J_k(R_1, R_2, ..., R_k)$ and refer to it as the {\em join result}. Set $n = \sum_{i=1}^k |R_i|$, i.e., the input size, and $\out = |\J_k(R_1, R_2, ..., R_k)|$, i.e., the output size.

\vgap

MSJ is a fundamental operation in spatial databases, which manage geometric entities such as land parcels, service areas, habitat zones, commercial districts, administrative boundaries, etc. It plays a crucial role in implementing the {\em filter-refinement mechanism} \cite{???}, the dominant approach in spatial databases for computing the overlay information of different entities. Specifically, a geometric entity, at the lowest level, is typically modeled as a polygon. Determining whether two entities overlap amounts to deciding if two polygons intersect, which can be very time consuming when the two polygons have complex boundaries. To alleviate the computation burden, a spatial database stores the {\em minimum bounding rectangle} (MBR) of each polygon $\P$, which is the smallest axis-parallel rectangle containing $\P$. Two polygons $\P_1$ and $\P_2$ can overlap with each other only if their MBRs $r_1$ and $r_2$ have a non-empty intersection.

\bibliographystyle{plainurl}% the mandatory bibstyle
\bibliography{ref}

\balance

%\end{sloppy}
\end{document}

