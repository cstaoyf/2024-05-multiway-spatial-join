\documentclass[sigconf]{acmart}
% \documentclass{...}
%\input{./def/yf-formatting}
\input{./def/yf-def}

%=================================
%Yufei's stuff
%\usepackage{amsmath}
\usepackage{balance}
%\usepackage{times}
\usepackage{microtype}

\def\vgap{\vspace{0mm}}
\def\extraspacing{\vspace{1mm} \noindent}
\def\figcapup{\vspace{-2mm}}
\def\figcapdown{\vspace{-4mm}}

\def\A{\mathcal{A}}
\def\B{\mathcal{B}}
\def\C{\mathcal{C}}
\def\E{\mathcal{E}}
\def\G{\mathcal{G}}
\def\I{\mathcal{I}}
\def\J{\mathcal{J}}
\def\II{\mathscr{I}}
\def\L{\mathcal{L}}
\def\P{\mathcal{P}}
\def\Q{\mathcal{Q}}
\def\R{\mathcal{R}}
\def\T{\mathcal{T}}
\def\U{\mathcal{U}}
\def\V{\mathcal{V}}
\def\X{\mathcal{X}}
\def\XX{\mathscr{X}}
\def\Y{\mathcal{Y}}
\def\YY{\mathscr{Y}}
\def\Z{\mathcal{Z}}

\def\out{\mathrm{OUT}}

\allowdisplaybreaks
%=================================

%====== from acm ======
\acmDOI{}
\acmISBN{}

\acmConference[]{...}{...}{...}
\acmYear{...}
\copyrightyear{}
\acmArticle{}
\acmPrice{}
%======================



\begin{document}
%\begin{sloppy}
    
\title{Optimal (Multiway) Spatial Joins}

% \author{}
% \affiliation{
% 	\institution{Chinese University of Hong Kong}
% 	\city{Hong Kong}
% 	\country{China}}	

\author{}


\begin{abstract}
    In a {\em spatial join}, we are given a constant number $k \geq 2$ of sets containing axis-parallel rectangles in a 2D space, denoted as $R_1, R_2, ..., R_k$. The objective is to report all $k$-tuples $(r_1, r_2, ..., r_k) \in R_1 \times R_2 \times ... \times R_k$ where the rectangles $r_1, r_2, ..., r_k$ have a non-empty intersection, i.e., $r_1 \cap r_2 \cap ... \cap r_k \neq \emptyset$. This problem holds significant importance in spatial databases and has been extensively studied for over two decades. We explain how to settle the problem in $O(n \log n + \out)$ time using $O(n)$ space --- regardless of the constant $k$ --- where $n = \sum_{i=1}^k |R_i|$ and $\out$ is the result size (i.e., the total number of $k$-tuples reported). The time complexity is asymptotically optimal in the class of comparison-based algorithms, to which our solution belongs. Our result significantly improves the state of the art, which is an algorithm with running time $O(n \log^{2k} n + \out)$.
\end{abstract}

\maketitle 

\section{Introduction} \label{sec:intro}

This paper revisits the {\em spatial join} (SJ) problem formulated as follows. Let $k \ge 2$ be a constant integer. In the {\em $k$-SJ} problem, the input comprises $k$ sets --- denoted as $R_1, R_2, ..., R_k$ --- of axis-parallel rectangles\footnote{A rectangle $r$ in 2D space is {\em axis-parallel} if it has the form $r = [x_1, x_2] \times [y_1, y_2]$.} in $\real^2$. The goal is to find all $k$-tuples $(r_1, r_2, ..., r_k)$ where
\myitems{
    \item $r_i \in R_i$ for each $i \in [1, k]$; and
    \item $r_1 \cap r_2 \cap ... \cap r_k \neq \emptyset$, namely, the $k$ rectangles $r_1, r_2, ..., r_k$ have a non-empty intersection.
}
We represent the set of $k$-tuples described above as $\J_k(R_1, R_2, ..., R_k)$, referred to as the {\em join result}. Set $n = \sum_{i=1}^k |R_i|$, i.e., the input size, and $\out = |\J_k(R_1, R_2, ..., R_k)|$, i.e., the output size.

\vgap

SJ is a fundamental operation in spatial databases (SDB), which manage {\em geometric entities} such as land parcels, service areas, habitat zones, commercial districts, administrative boundaries, etc. It plays a crucial role in implementing the {\em filter-refinement mechanism}, which is the dominant approach for computing overlay information in an SDB. To explain this mechanism, first note that a geometric entity is typically modeled as a polygon. Determining whether two entities overlap amounts to deciding if two polygons intersect, which can be exceedingly expensive when the polygons have complex boundaries. To mitigate the issue, an SDB stores, for each polygon $g$, its {\em minimum bounding rectangle} (MBR) defined as the smallest axis-parallel rectangle enclosing $g$; this way, each set $G$ of geometric entities spawns a set $R$ of MBRs. Consider $k$ sets of geometric entities $G_1, G_2, ..., G_k$, and the corresponding sets of MBRs $R_1, R_2, ..., R_k$. To compute overlays from $G_1, G_2, ..., G_k$, filter-refinement first executes (i) ``a filter step'', which performs an SJ to obtain $\J_k(R_1, R_2, ..., R_k)$, and (ii) a ``refinement step'', which, for each $(r_1, r_2, ..., r_k) \in \J_k(R_1, R_2, ..., R_k)$, examines if $P_1, P_2, ..., P_k$ indeed have a non-empty intersection, where $P_i$ ($i \in [1, k]$) is the entity in $G_i$ whose MBR is $r_i$.

%Two polygons $P_1$ and $P_2$ can overlap with each other only if their MBRs $r_1$ and $r_2$ intersect.

\extraspacing {\bf Math Conventions.} For any integer $x \ge 1$, we use $[x]$ to represent the set $\set{1, 2, ..., x}$. Every mention of the word ``rectangle'' henceforth will refer to an axis-parallel rectangle. All logarithms have base 2 by default.

\subsection{Previous Results} \label{sec:intro:prev}

SJs have been extensively studied in the database-system community, leading to the development of numerous methods that, although lacking strong theoretical guarantees, exhibit good empirical performance in real-world applications. We refer interested readers to \cite{apr+00,bks93,gcn+13,js07,ks97,lr94,lr96,mp98,mp01,mp03,pd96,pmt99} as entry points into the literature.

\vgap

From the perspective of theory, SJs are best understood when $k = 2$, i.e., the {\em pairwise} scenario, where it is folklore that the problem can be solved in $O(n \log n + \out)$ time using $O(n)$ space (e.g., by planesweep \cite{bcko08}). However, the problem becomes significantly more challenging for $k \ge 3$, i.e., the {\em multiway} scenario. All the solutions developed  before 2022 (see \cite{gcn+13,mp98,mp01,pmt99} and the references therein) suffer from a worst-case time complexity of $O(n^k)$, offering essentially no improvement over the naive method that enumerates the entire cartesian product $R_1 \times R_2 \times ... \times R_k$.


\begin{table*} 
    \begin{tabular}{c|c|c|c|c} 
        $\bm{k}$ & {\bf method} & {\bf runtime} & {\bf space} & {\bf remark} \\
        \hline\hline 
        2 & folklore & $O(n \log n + \out)$ & $O(n)$ & comparison-based optimal \\ 
        \hline
        $\ge 3$ & before 2022 & $O(n^k)$ & - & \\
        $\ge 3$ & \cite{ty22} & $O((n + \out) \cdot \polylog n)$ & $O(n \polylog n)$ & \\
        $\ge 3$ & \cite{kcko22} & $O(n \log^{2k} n + \out)$ & $O(n \log n)$ & \\
        \hline
        $\ge 3$ & ours & $O(n \log n + \out)$ & $O(n)$ & comparison-based optimal
    \end{tabular}
    
    \vspace{3mm}
    \caption{Comparison of results on the $k$-SJ problem for a constant $k$}
    \label{tab:results-com}
\end{table*}

\vgap

Year 2022 witnessed two independent works \cite{ty22,kcko22} that, although not tackling $k$-SJ directly, imply provably fast $k$-SJ algorithms. Specifically, in \cite{ty22}, Tao and Yi studied several variants of ``interval intersection joins'' under updates. Most relevant to our context is the variant where the input includes, for each $i \in [k]$, a set $\I_i$ of 1D intervals in $\real$, and the join result comprises all $k$-tuples $(I_1,$ $I_2,$ $..., I_k) \in \I_1 \times \I_2 \times ... \times \I_k$ with $\bigcap_{i=1}^k I_i \neq \emptyset$. The objective is to design a data structure, which, given the insertion (resp., deletion) of an interval in one of the $k$ sets, can identify all the newly-appearing (resp., disappearing) $k$-tuples in the join result in $O((1+\Delta) \cdot \polylog n)$ time, where $n = \sum_{i=1}^k |\I_i|$ and $\Delta$ is the number of such $k$-tuples. Tao and Yi \cite{ty22} presented a structure of $O(n \polylog n)$ space achieving the purpose. Combining their structure with planesweep, one can obtain an algorithm for solving the $k$-SJ problem in $O((n + \out) \cdot \polylog n)$ time.

\vgap

In \cite{kcko22}, Khamis et al.\ investigated a type of joins that extends the conventional equi-join in two ways. First, each attribute value in a relation is an interval (rather than a real value); second, each equality predicate in equi-join is replaced with a ``non-empty intersection'' predicate on the attributes involved. The $k$-SJ problem can be converted to solving a join defined next under the framework of \cite{kcko22}. For each $i \in [k]$, define $R_i$ as a relation over two attributes $X$ and $Y$. For each tuple $\bm{u} \in R_i$, its values $\bm{u}(X)$ and $\bm{u}(Y)$ on the two attributes are both intervals (effectively defining a rectangle). The objective is to output all $k$-tuples $(\bm{u}_1, \bm{u}_2, ..., \bm{u}_k) \in R_1 \times R_2 \times ... \times R_k$ satisfying $\bigcap_{i=1}^k \bm{u}_i(X) \ne \emptyset$ and $\bigcap_{i=1}^k \bm{u}_i(Y) \ne \emptyset$. It is clear that there is one-one correspondence between the result of this join and that of k-SJ. Khamis et al.\ \cite{kcko22} developed an algorithm that can process the join  in $O(n \log^{2k} n + \out)$ time. Their algorithm requires $O(n \log n)$ space.

\vgap 

It is worth noting that $\Omega(n \log n)$ is a lower bound on the runtime of any comparison-based algorithm solving the $k$-SJ problem, even for $k = 2$. This can be established via a reduction (presented in Appendix~\ref{app:lb} for self-containment) from the {\em element distinctness} (ED) problem, where we are given $n$ real values $e_1, e_2, ..., e_n$ and need to decide whether there are distinct $i, j \in [n]$ satisfying $e_i = e_j$. The ED problem demands $\Omega(n \log n)$ comparisons to solve \cite{dl79}.

\todo{Rahul's result? No. Do it in later sections}

\subsection{Our Results} \label{sec:intro:ours} 

In this paper, we show that the $k$-SJ problem can be settled by a comparison-based algorithm that runs in $O(n \log n + \out)$ time and uses  $O(n)$ space, regardless of the constant $k$. The time complexity is asymptotically optimal and significantly improves the current state of the art \cite{kcko22}.

\vgap 

In terms of techniques, our primary contribution is the revelation of a new property on the problem's mathematical structure. Fix any $k \ge 3$ and an {\em arbitrary} algorithm $\A$ for the $(k-1)$-SJ problem. For any $n \ge 1$, define function $F_{k-1}(n, \out)$ to return the worst-case running time of $\A$ on any instance of the $(k-1)$-SJ problem having input size at most $n$ and output size at most $\out$. Our main technical contribution is to establish:

\begin{theorem} \label{thm:main-recur}
    For any $n \ge 1$, equipped with the algorithm $\A$ as described above, the $k$-SJ problem can be solved in
    \myeqn{
        O(k^3) \cdot \big( F_{k-1}(n, \out) + n \log n + \out \big)
        \label{eqn:main:reccurrence}
    }
    time using asymptotically the same space as $\A$.
\end{theorem}

The theorem implies the existence of a recursive nature of $k$-SJ. Indeed, we will see that the $k$-SJ can be converted to $O(k^3)$ instances of the $(k-1)$-SJ problem --- all of which have input size at most $n$ and output size at most $\out$ --- plus an additional processing cost of $O(n \log n + \out)$. Furthermore, if $\A$ is comparison-based, so is the $k$-SJ algorithm that ensues. For 2-SJ, we can set $\A$ simply to the ``folklore algorithm'' mentioned in Section~\ref{sec:intro:prev}, which ensures $F_2(n, \out) = O(n \log n + \out)$ and uses linear space. Combining this with \eqref{eqn:main:reccurrence} yields a recurrence that relates the time complexity of $k$-SJ to that of $(k-1)$-SJ. Solving the recurrence yields:

\begin{theorem} \label{thm:main-alg}
    For $k \ge 3$, we can settle $k$-SJ with a comparison-based algorithm in $O( c^k (k!)^3 \cdot (n \log n + \out))$ time and $O(k \cdot n)$ space, where $c > 1$ is a positive constant.
\end{theorem}

When $k = O(1)$, the time and space complexities become $O(n \log n + \out)$ and $O(n)$, as promised.

\vgap 

Now that Theorem~\ref{thm:main-alg} offers a satisfactory $k$-SJ result for $k = O(1)$ in 2D space, it is natural to wonder whether the constraint on dimensionality 2 is necessary. The answer, interestingly, is ``yes'' as far as $k \ge 3$ is concerned, subject to the absence of breakthroughs on a classical problem in graph theory. Specifically, if the 3D version of the 3-SJ problem (which we will formally define in Appendix~\ref{app:lb-cond}) could be solved in $O(n \polylog n + \out)$ time, we would obtain an algorithm that detects the presence of a triangle (i.e., 3-clique) in a graph of $m$ edges in $O(m \polylog m)$ time, which would make a truly surprising breakthrough because the state of the art needs $O(m^{1.41})$ time \cite{ayz97}. This reduction can be inferred from an argument in \cite{kcko22} used to prove a more generic result. We simplify the argument for 3D 3-SJ and present the full reduction in Appendix~\ref{app:lb-cond}.

\bibliographystyle{plainurl}% the mandatory bibstyle
\bibliography{ref}

\balance

\appendix 

\section{An $\bm{\Omega(n \log n)}$ Lower Bound for 2-SJ} \label{app:lb}

\section{Hardness of 3-SJ in 3D Space} \label{app:lb-cond}

%\end{sloppy}
\end{document}

