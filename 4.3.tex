<<<<<<< HEAD
=======
\noindent \underline{\em Proof of Statement (1).}
We will map each $r^*\in \bigcup_{i = 1}^{k-2} R_i^*$ to a unique tuple $\bm{t}\in \J(R_1',...,R_{k-2}')$ such that $\dcross_H(\bm{t}) = \cross_H(r^*)$. Then, by Lemma~\ref{lmm:hv:type1:properties} (3), we have $\sum_{i\in[k-2]}\sum_{r^*\in R_i^*} |\cross_H(r^*)| \leq \sum_{\bm{t}\in \J(R_1',...,R_{k-2}')}|\dcross_H(\bm{t})|\leq \out$.

\vgap 

Now we explain the mapping. Let $r^*$ be the top-sliced rectangle of $r'$. By how $\maxtop(r')$ is defined, we can identify a tuple $\bm{t}\in \J(R_1',...,R_{k-2}')$ such that 
\begin{itemize}
    \item $\gbot(\bm{t}) = r'$,
    \item $\maxtop(r') = \ytop(B_{\bm{t}})$.
\end{itemize}
Let $r' = [x_1,x_2]\times [y_1,y_2]$, and $r^* = [x_1,x_2]\times [y_1,\maxtop(r')]$. Since $\gbot(\bm{t}) = r'$, we can represent $B_{\bm{t}}$ as $[\xleft(B_{\bm{t}}), \xright(B_{\bm{t}})]\times [y_1, \maxtop(r')]$.
We will prove that a segment $h\in \cross_H(r^*)$ if and only if $h\in \dcross_H(\bm{t})$. 

\vgap 

We show the ``if'' direction as follows. Let $h\in \cross_H(r^*)$. First, we prove that $h$ crosses $r'$. Since $h$ crosses $r^*$, $h \cap r^* \neq \emptyset$. As $r^* \subseteq r'$, $r'\cap h \neq \emptyset$. It suffices to show that $r'$ does not containe any endpoints of $h$.
Suppose to the contrary that $r'$ contains at least an endpoint of $h$. W.l.o.g, assume that the left endpoint of $h$ is in $r'$.
Then, $\xleft(h)\in[x_1,x_2]$. Since $h\cap r^*\neq \emptyset$, the y-coordinate of $h$ is in $[y_1,\maxtop(r')]$. Since $r^*$ does not contain the any endpoint of $h$, we have $\xleft(h)<x_1<x_2<\xright(h)$, 
which is a contradiction. Therefore, $h$ crosses $r'$.  

\vgap 

Then, we prove that $h$ crosses $B_{\bm{t}}$. Since $h$ crosses $r'$ and $B_{\bm{t}}\subseteq r'$, $B_{\bm{t}}$ does not contain any endpoints of $h$. Let the y-coordinate of $h$ be $y$. Let $p = (\xleft(B_{\bm{t}}), y)$. We will prove that $p \in h$ and $p \in B_{\bm{t}}$, which means $h\cap B_{\bm{t}} \neq \emptyset$ and $h$ crosses $B_{\bm{t}}$. Since $\gbot(\bm{t}) = r'$, the bottom segment of $B_{\bm{t}}$ is contained in $r'$. Therefore,  
$\xleft(B_{\bm{t}})\in [x_1,x_2]$. We proved in the last paragraph that $\xleft(h)<x_1<x_2<\xright(h)$, so  $\xleft(B_{\bm{t}}) \in (\xleft(h),\xright(h))$, and $p\in h$. To prove that $p\in B_{\bm{t}}$, it suffices to show that $y$ is in the y-range of $B_{\bm{t}}$, i.e. $y \in [\ybot(B_{\bm{t}}), \ytop(B_{\bm{t}})]$.
 Since $h$ intersects $r^*$, $y \in [y_1 , \maxtop(r')]$, where $\maxtop(r') = \ytop(B_{\bm{t}})$. Since $\gbot(\bm{t}) = r'$, $y_1 = \ybot(B_{\bm{t}})$. Hence, $y\in [\ybot(B_{\bm{t}}), \ytop(B_{\bm{t}})]$, and $p\in B_{\bm{t}}$.  

\vgap 

Conversely, let $h\in \dcross_H(\bm{t})$, which means that $h$ crosses $r'$ and $B_{\bm{t}}$. Thus, $r'$ does not contain any endpoints of $h$. Since $r^*\subseteq r'$, $r^*$ does not contain any endpoints of $h$ neither. 
<<<<<<< HEAD
Next, we will prove that $B_{\bm{t}} \subseteq r^*$. Since $h \cap B_{\bm{t}}\neq \emptyset$, this statement leads to $h \cap r^* \neq \emptyset$ and $h$ crosses $r^*$. Recall that $r^* = [x_1,x_2], [y_1, \maxtop(r')]$, and $B_{\bm{t}} = [\xleft(B_{\bm{t}}), \xright(B_{\bm{t}})]\times [y_1,\maxtop(r')]$. Since $B_{\bm{t}}$ is a subset of $r'$, we have $[\xleft(B_{\bm{t}}), \xright(B_{\bm{t}})] \subseteq [x_1,x_2]$. Therefore, $B_{\bm{t}} \subseteq r^*$, and $h\in \cross_H(r^*)$. 
=======
Next, we prove that $B_{\bm{t}} \subseteq r^*$. Since $h \cap B_{\bm{t}}\neq \emptyset$, this statement leads to $h \cap r^* \neq \emptyset$ and $h$ crosses $r^*$. Recall that $r^* = [x_1,x_2], [y_1, \maxtop(r')]$, and $B_{\bm{t}} = [\xleft(B_{\bm{t}}), \xright(B_{\bm{t}})]\times [y_1,\maxtop(r')]$. Since $B_{\bm{t}}$ is a subset of $r'$, we have $[\xleft(B_{\bm{t}}), \xright(B_{\bm{t}})] \subseteq [x_1,x_2]$. Therefore, $B_{\bm{t}} \subseteq r^*$, and $h\in \cross_H(r^*)$. 
>>>>>>> 441351770ea227b2ce95de038c2a41e5f65280ad

\vgap 

It remains to show that we will not map two distinct rectangle $r_1^*, r_2^*\in \bigcup_{i = 1}^{k-2}R_i^*$ to the same tuple $\bm{t}\in \J(R_1',...,R_{k-2}')$. Suppose to the contrary that $r_1^*, r_2^*$ are mapped to the same tuple $\bm{t}$. Let $r_1^*, r_2^*$ be the top-sliced rectangles of $r_1'$ and $r_2'$ respectively. Since $r_1^*$ and $r_2^*$ are disinct, $r_1'$ is not the same as $r_2'$. By our mapping, $\gbot(B_{\bm{t}})$ is $r_1'$ and $r_2'$. Since $\gbot(B_{\bm{t}})$ is unique for each tuple $\bm{t}$, we have a contradiction. Therefore, each rectangle $r^*\in \bigcup_{i = 1}^{k-2}R_i^*$ is mapped to a unique tuple $\bm{t}\in \J(R_1',...,R_{k-2}')$.

>>>>>>> fc9fcecf3343c248a654ce6af7c44ce99af1e489
\noindent \underline{\em Proof of Statement (2).} 
