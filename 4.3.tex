First, we prove (1). For any $i \in [k-2]$, $r_i^*\in R_i^*$, and $h \in \cross_H(r_i^*)$, we will identify a unique tuple $\bm{t} \in \J(R_1,...,R_{k-2},H,V)$ such that $\bm{t}[k-1] = h$, and $\bm{t}[i]$ is the full rectangle of $r_i'$, whose top-slice is $r_i^*$. Let $\maxtop(r_i') = \ytop(B_\bm{t'})$ for some tuple $\bm{t'} \in \J(R_1',...,R_{k-2}')$. Let $\bm{t'} = (r'_1,...,r'_{k-2})$, and let $r_i$ be the full rectangle of $r_i'$ for $i \in [k-2]$. Since $h$ crosses $r_i^*$, and $B_\bm{t'} \subseteq r_i^*$, $h$ crosses $B_{\bm{t'}}$. Let $v$ be the segment $v_\vdash$ of $\gleft(\bm{t'})$. The intersection of $h$ and $v$ is a point on the left boundary of $B_{\bm{t'}}$, which means that $h\cap v \in B_{\bm{t'}} \subseteq \cap_{i = 1}^{k-2}r_i$. Therefore, $\bm{t} = (r_1,...,r_{k-2},h,v)\in \J(R_1,...,R_{k-2},H,V)$ is the desired tuple. As a result, $\sum_{i \in [k-2]}\sum_{r_i^*\in R_i^*} |\cross_H(r_i^*)| \leq \out$.