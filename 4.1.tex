
 We will prove by associating each tuple $\bm{t} \in \J(R_1',...,R_{k-2}')$ with a unique tuple in the H-V $k$-SJ result set $\J(R_1,...,R_{k-2},H,V)$. More specifically, let $r_i$ be the full rectangle of $\bm{t}[i]$ for $i\in[k-2]$. Let $\bm{t}[l] = \gleft(\bm{t})$ and $\bm{t}[b] = \gbot(\bm{t})$, where $l,b\in[k-2]$. Let $h$ be the segment $h_\bot$ for the rectangle $r_b$, and $v$ be the segment $v_\vdash$ for the rectangle $r_l$. The intersection of $h$ and $v$ is the bottom-left coner of the intersection area $\bigcap_{i = 1}^{k-2}\bm{t}[i]$. Hence, $h\cap v\cap \bigcap_{i = 1}^{k-2}r_i \neq \emptyset$ and the tuple $(r_1,...,r_{k-2},h,v)$ is a result tuple in $\J(R_1,...,R_{k-2},H,V)$. The result tuple is unique because for any two distinct tuples $\bm{t_1}$ and $\bm{t_2}$ in $\J(R_1',...,R_{k-2}')$, there exists an integer $i \in [k-2]$ s.t. the full rectangles of $\bm{t_1}[i]$ and $\bm{t_2}[i]$ are different. Therefore, $|\J(R_1',...,R_{k-2}')| \leq |\J(R_1,...,R_{k-2})| = \out$.
