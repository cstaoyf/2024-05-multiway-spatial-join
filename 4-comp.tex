We will first compute $\maxtop(r')$ for each rectangle $r' \in \bigcup_{i = 1}^{k-2}R_{i}'$ by scanning the set $\J(R_1',...,R_{k-2}')$. Initially, we set $\maxtop(r') = -\infty$ for each $r' \in \bigcup_{i = 1}^{k-2}R_{i}'$. Then, for each tuple $\bm{t} = (r_1',...,r_{k-2}') \in \J(R_1',...,R_{k-2}')$, we identify the rectangle $\gbot(\bm{t})$ in $O(k)$ time and 
 update $\maxtop(\gbot(\bm{t}))$ to the maximum between its current value and $\ytop(B_{\bm{t}})$. The time complexity so far is $O(n + k\cdot \out)$.
Finally, we can construct the set $R_i^*$ for every $i \in [k-2]$ by collecting the top-sliced rectangles of all rectangles $r_i' \in R_i'$ satisfying that $\maxtop(r_i')\neq -\infty$. This step takes $O(n)$ additional time. The overall time complexity is $O(n + k\cdot \out)$.